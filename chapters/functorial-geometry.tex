\NewDocumentCommand\Inv{sm}{\IfBooleanTF{#1}{\brc{#2}}{#2}^{-1}}
\NewDocumentCommand\Aff{}{\mathbf{Aff}}
\NewDocumentCommand\Sch{}{\mathbf{Sch}}
\NewDocumentCommand\AffFP{}{\mathbf{Aff}_{\textit{f.p.}}}
\NewDocumentCommand\Ring{}{\mathbf{Ring}}
\NewDocumentCommand\LocRing{}{\mathbf{Ring}_{\mathrm{Loc}}}
\NewDocumentCommand\RingFP{}{\mindelim{1}\mathbf{Ring}_{\textit{f.p.}}}
\NewDocumentCommand\RingedTOP{}{\mathbf{RingedTop}}
\NewDocumentCommand\LocRingedTOP{}{\mathbf{RingedTop}_{\mathrm{Loc}}}
\NewDocumentCommand\Spec{g}{\operatorname{Spec}\IfValueT{#1}{#1}}
\NewDocumentCommand\Zar{m}{\mathrm{Zar}\prn{#1}}
\NewDocumentCommand\Str{G{\bullet}}{\mathcal{O}_{#1}}
\NewDocumentCommand\GS{g}{\Gamma\IfValueT{#1}{\prn{#1}}}
\NewDocumentCommand\Ind{m}{\mathrm{Ind}\prn{#1}}
\NewDocumentCommand\Pts{m}{\mathrm{Pt}\prn{#1}}
\NewDocumentCommand\UU{g}{\mathscr{U}\IfValueT{#1}{_{#1}}}
\NewDocumentCommand\VV{g}{\mathscr{V}\IfValueT{#1}{_{#1}}}
\NewDocumentCommand\WW{g}{\mathscr{W}\IfValueT{#1}{_{#1}}}
\NewDocumentCommand\FunSp{}{\ECat}
\NewDocumentCommand\FunPreSp{}{\FunSp_{\mathit{pre}}}

\nocite{eisenbud-harris:2000}

\chapter{Functorial geometry}
\minitoc



I claim no originality for these notes! They are drawing together a number of
ideas on the functorial formulation of algebraic geometry, due to a number of
authors.\footnote{See for instance \citet{lurie:dag:5,cole:spectra:reprint,hakim:1972,sga:4,demazure-gabriel:1980,blechschmidt:2017}.} The aim of these notes is to teach myself some geometry in a language I
understand --- consequently, the presentation goes in a more topos theoretic
direction than most expositions.

I am testing the hypothesis that if one allows oneself the ``forbidden
knowledge'' of approximately two semesters of topos theory, a number of things
become easier and less technical. These notes owe a lot in particular to
\citet{blechschmidt:2017,johnstone:1977,lurie:dag:5}.

\section{Category theoretic preliminaries}

\subsection{Universes}

\begin{para}
  We follow the convention of Grothendieck universes, and in particular assume
  that every set is contained in a universe $\UU$. Each universe $\UU$ is assumed
  to correspond to a strongly inaccessible cardinal $\alpha>\omega$, and is
  consequently closed under dependent sum, dependent product, and power sets.
\end{para}


\begin{para}
  %
  We adopt a vernacular approach to the relative sizes of sets, rooted at an
  arbitrary strongly inaccessible cardinal.
  %
  \begin{enumerate}
    \item $\UU$ is the (normal-sized) universe of small sets
    \item $\VV$ is the (large) universe of normal-sized sets
    \item $\WW$ is the (very large) universe of large sets
  \end{enumerate}
\end{para}

\begin{para}
  %
  A \emph{$\UU$-category} is a category $\CCat$ whose collection of objects is unconstrained,
  and whose hom-sets $\Hom[\CCat]{A}{B}$ are all contained in $\UU$.
  %
\end{para}

\begin{para}
  A \emph{$\UU$-small category} is a $\UU$-category whose collection of objects $\Ob{\CCat}$ is contained in $\UU$.
\end{para}

\begin{para}
  We fix the following notations for categories of sets and categories:
  \begin{enumerate}
    \item The category of $\UU$-small sets is written $\SET[\UU]$.
    \item The category of $\UU$-categories is written $\CAT[\UU]$. 
    \item The category of $\UU$-small categories is written $\SmCAT[\UU]$.

    \item Given any category $\CCat$, the category of $\UU$-presheaves on
      $\CCat$ is written $\Psh[\UU]{\CCat} = \Hom{\OpCat{\CCat}}{\SET[\UU]}$.
  \end{enumerate}
\end{para}

\subsection{Topoi and logoi}

We share some notations and conventions with \citet{anel-joyal:2019}.

\begin{para}
  %
  Let $\CCat$ be a category; we will write $\AlxTop{\CCat}$ for the \emph{Alexandrov topos}
  defined by the equation $\Sh{\AlxTop{\CCat}} = \CoPsh{\CCat} =
  \Psh{\OpCat{\CCat}}$. A good intuition is that $\CCat$ is the category of
  finitely presented models for a finite limit theory $\ThCat$, and then
  $\AlxTop{\CCat}$ is the classifying topos or moduli space of $\ThCat$-models.

  The category of points of $\AlxTop{\CCat}$ is $\Ind{\CCat}$; in case $\CCat$
  is the category of finitely presented $\ThCat$-models, then $\Ind{\CCat}$ is
  the category of \emph{all} $\ThCat$-models, being the free completion under
  filtered colimits.
  %
\end{para}

\begin{para}
  %
  Let $\CCat$ be a category; we will write $\PrTop{\CCat}=
  \AlxTop{\OpCat{\CCat}}$ for the topos defined by the equation
  $\Sh{\PrTop{\CCat}} = \Psh{\CCat}$. Here, the intuition is best when $\CCat =
  \ThCat$ is a finite limit theory; then $\PrTop{\ThCat}$ is the classifying
  topos of $\ThCat$-models.
  %
\end{para}

\begin{para}
  %
  The important idea hiding in the intuitions above is the famous
  Gabriel--Ulmer duality between a finite limit theory and its category of its
  finitely presented models~\citep{gabriel-ulmer:1971}, itself a variant of the
  Lawvere duality between a finite product theory and its category of free
  finitely generated models~\citep{lawvere:thesis}.
  %
\end{para}

\subsection{Stalkwise conditions}

\begin{para}
  %
  In classic presentations of scheme theory, many conditions on sheaves of
  algebras are tested stalkwise; this is only possible because most topoi under
  consideration in geometry have enough points!
\end{para}

\begin{para}
  %
  Let $\XTop$ be a topos; we denote by $\Pts{\XTop}$ the category of points
  $\Hom[\TOP]{\Pt}{\XTop}$. Given a point $x:\Pts{\XTop}$ and a sheaf
  $E:\Sh{\XTop}$, the \emph{stalk} $E_x : \SET$ of $E$ at $x$ is the inverse
  image $x^*E$.
  %
\end{para}
 
\begin{para}
  %
  Let $\XTop$ be a topos; $\XTop$ is said to \emph{have enough points} when any
  morphism of $\Mor[f]{E}{F} : \Sh{\XTop}$ is an isomorphism if for every point $x: \Pts{\XTop}$ the function on stalks $\Mor[f_x]{E_x}{F_x}$ is an isomorphism.
  %
\end{para}


\begin{para}
  %
  Rather than stating conditions relative to the stalks, we prefer to state
  them \emph{directly} in terms of the internal logic of each topos. This
  approach is more robust, and works even when a topos doesn't have enough
  points.
  %
\end{para}

\section{Commutative algebra}

In these notes, all rings are assumed commutative with unit and homomorphisms
of rings are assumed to preserve $1$.

\begin{para}
  %
  A ring is called \emph{local} it has a unique maximal ideal. When $A,B$ are
  local rings, a homomorphism $\Mor[f]{A}{B}$ is local when it reflects
  invertibility, i.e.\ for $a\in A$ if $f(a)$ is a unit then $a$ is a unit.
  %
\end{para}

\begin{para}
  %
  We will write $\Ring$ for the category of rings and ring homomorphisms and
  $\LocRing$ for the category of local rings and local ring homomorphisms.
  %
\end{para}

\begin{namespace}{partition-of-unity}
  \begin{para}[Partition of unity]\nslabel{para}
    %
    Let $A$ be a ring and let $f_i$ be a finite set of elements of $A$ that generate the unit ideal. Then
    the following diagram
    is an equalizer in $\SET$:
    \[
      \begin{tikzpicture}[diagram]
        \node (0) {$A$};
        \node (1) [right = of 0] {$A\brk{\Inv{f_i}}$};
        \node (2) [right = 2.5cm of 1] {$\Prod{i,j}{A\brk{\Inv{f_{i,j}}}}$};
        \path[->] (0) edge (1);
        \path[->,shift left =.1cm] (1) edge (2);
        \path[->,shift right =.1cm] (1) edge (2);
      \end{tikzpicture}
    \]

    \begin{proof}
      Suppose we have a matching family of elements $a_i\in A\brk{\Inv{f_i}}$
      agreeing on the overlaps. Then each $a_i$ can be written as a formal fraction
      ${\tilde{a}_i}/{f_i^m}$ for a uniform sufficiently large power $m$. The
      condition that these match on the overlapped localizations implies that for
      some sufficiently large $k$ we have $a_if_{ij}^k = a_j f_{ij}^k$, writing $f_{ij}$ for $f_if_j$.
      %
      Hence, replacing $a_i$ by $\tilde{a_i}/f_i^m$ and canceling on both sides, we have:
      \begin{equation}\nslabel{eqn:rewrite}
        \tilde{a}_i f_j^m f_{ij}^k = \tilde{a}_j f_i^m f_{ij}^k
      \end{equation}

      Because the $f_i$ generate the unit ideal, so do $f_i^{m+k}$; hence there
      exist $c_i$ such that $1 = \Sum{i}{c_i f_i^{m+k}}$. For the amalgamation, we
      choose $a = \Sum{i}{c_i\tilde{a}_i f_i^k}$; we must check that $a = a_j$ in
      each $A\brk{\Inv{f_j}}$. First we observe that each $a_j$ can be massaged into a
      form where \nscref{eqn:rewrite} applies:
      %
      \begin{equation}\nslabel{eqn:massage}
        a_j = \frac{\tilde{a}_j}{f_j^m} 
        = \frac{\tilde{a}_j f_j^{k}}{f_j^{m+k}}
        = \frac{\tilde{a}_j f_j^{k}}{f_j^{m+k}}\Sum{i}{c_if_i^{m+k}}
        =
        \Sum*{i}{
          \frac{
            c_i \tilde{a}_j f_j^k f_i^{m+k}
          }{
            f_j^{m+k}
          }
        }
        =
        \Sum*{i}{
          \frac{
            c_i \tilde{a}_j f_i^m f_{ij}^k
          }{
            f_j^{m+k}
          }
        }
      \end{equation}

      Hence, we can see that $a$ restricts to $a_j$ as desired:
      \begin{align*}
        a_j
        &=
        \Sum*{i}{
          \frac{
            c_i \tilde{a}_i f_j^m f_{ij}^k
          }{
            f_j^{m+k}
          }
        }
        &&\text{by \nscref{eqn:rewrite,eqn:massage}}
        \\
        &=
        \Sum*{i}{
          \frac{
            c_i \tilde{a}_i f_{ij}^k
          }{
            f_j^{k}
          }
        }
        &&\text{cancel $f_j^m$}
        \\
        &=
        \Sum{i}{
          c_i \tilde{a}_i f_{i}^k
        }
        &&\text{cancel $f_j^k$}
        \\
        &= a
        &&\text{by definition}
      \end{align*}

      It remains to show that $a$ is unique with this property. Suppose there
      were some other $a'$ that extended each $a_i$; without loss of generality,
      we may consider the case that $a'=0\in A$ and hence each $a=a'=a_i=0\in
      A\brk{\Inv{f_i}}$ (since each restriction map is a ring homorphism), and we want to
      show that $a=0\in A$. By the aforementioned equation in $A\brk{\Inv{f_i}}$,
      there exists $m$ large enough to make $f_i^m a = f_i^m a' = 0 \in A$ for
      all $i$. But we may find $d_i$ such that $1 = \Sum{i}{d_i f_i^m}$, hence
      $a = a\Sum{i}{d_i f_i^m} = \Sum{i}{d_i f_i^m a} = 0$.
    \end{proof}
  \end{para} 
\end{namespace}



\subsection{Classifying topoi}

\begin{para}
  %
  The theories of rings and local rings are both geometrical; consequently,
  they both have classifying topoi, respectively $\brk{\Ring},\brk{\LocRing}$.
  These topoi can be constructed manually, but what really matters is their
  universal property!

  We will write $\SET\brk{\Ring} = \Sh{\brk{\Ring}}$ for the corresponding
  category of sheaves, to emphasize the sense in which a classifying topos
  freely adds a generic model of a given theory to the background set theory.
  %
\end{para}

\begin{para}
  %
  The classifying topos of rings is the Alexandrov topos $\brk{\Ring} =
  \AlxTop{\RingFP}$; on the other hand, the classifying topos of local rings
  has a more complex and famous construction known as the \emph{Zariski topos}.
  In particular, $\SET\brk{\LocRing}$ is a localization of $\SET\brk{\Ring}$
  generated by the \emph{Zariski topology}.
  %
\end{para}

% \section{Affine schemes}
% 
% \begin{para}
%   %
%   An \emph{affine scheme} is the formal dual of a commutative ring, i.e.\ an
%   object of the category $\Aff = \OpCat{\Ring}$. 
%  % When $A$ is a commutative
%  % ring, we will write $\Spec{A}$ for the corresponding affine scheme;
%  % conversely, when $X$ is an affine scheme, we write $\Str{X}$ for the
%  % corresponding commutative ring.
%   %
% \end{para}
% 
% \begin{para}
%   A \emph{distinguished open immersion} of affine schemes is the dual to a
%   localization map $\Mor{A}{A\brk{\Inv{f}}}$ of commutative rings for some $f \in A$.
%   %in which $Y \cong \Spec{\Str{X}\brk{f^{-1}}}$ for a function $f \in
%   %\Str{X}$. 
% \end{para}
% 
% %\begin{definition}
% %  A finitely presented affine scheme is an object of the category $\AffFP = \OpCat{\RingFP}$.
% %\end{definition}
% 

\section{Geometric schemes, affine and general}

We first present the view of ``geometric'' schemes as locally ringed topoi,
before bootstrapping a notion of functorial scheme.

\subsection{Locally ringed topoi}

\begin{para}
  %
  A \emph{ringed topos} is a pair $\prn{\XTop,\Str{\XTop}}$ where $\XTop$ is a
  topos and $\Str{\XTop}$ is a \emph{sheaf of rings} in on $\XTop$, i.e.\
  morphism of topoi $\Mor{\XTop}{\brk{\Ring}}$.
%
  We will frequently abuse notation by writing $\XTop$ for the pair
  $\prn{\XTop,\Str{\XTop}}$.
\end{para}

\begin{para}
  The correct notion of morphism between ringed topoi is given by the \emph{lax
  slice} 2-category $\RingedTOP = \LaxSl{\TOP}{\brk{\Ring}}$ whose objects are
  evidently the ringed topoi.
  %
  Consequently, a 1-morphism of ringed topoi
  $\Mor{\prn{\XTop,\Str{\XTop}}}{\prn{\YTop,\Str{\YTop}}}$ is a morphism of
  topoi $\Mor[f]{\XTop}{\YTop}$ together with a morphism of ring objects
  $\Mor[f^\sharp]{f^*\Str{\YTop}}{\Str{\XTop}}$ called the \emph{comorphism}.
  %
\end{para}

\begin{para}
  From the algebraic perspective, the structure sheaf
  $\Mor[\Str{\XTop}]{\XTop}{\brk{\Ring}}$ is a lex and cocontinuous functor
  $\Mor[\Str{\XTop}^*]{\SET\brk{\Ring}}{\Sh{\XTop}}$. Precomposing with the
  Yoneda embedding, one obtains a functor $\Mor{\OpCat{\RingFP}}{\Sh{\XTop}}$
  that we shall also denote by $\Str{\XTop}^*$.

  Every finitely presented commutative ring can be thought of as a ``context''
  in the language of rings, where we pay special attention to two distinguished forms of context extension:
  freely adding a new element, and adding an inverse to an existing element.
%
  Letting $A = \mathbb{Z}\brk{x}$ be the free ring generated by a single element, it is
  correct to think of $\Str{\XTop}^*A$ as the sheaf collection of elements of
  $\Str{\XTop}$; likewise, it is correct pitchfork think of
  $\Str{\XTop}^*\prn{A\brk{\Inv{x}}}$ as the sheaf of units of $\Str{\XTop}$.
\end{para}

\begin{para}
  %
  A locally ringed topos is a ringed topos $\XTop$ such that $\Str{\XTop}$ is a
  local ring, i.e.\ a ring object such that ``$\Str{\XTop}$ has a unique
  maximal ideal'' holds in the internal logic of $\Sh{\XTop}$.
%
  A morphism of locally ringed topoi is a morphism of ringed topoi
  $\Mor[\prn{f,f^\sharp}]{\XTop}{\YTop}$ such that the induced naturality
  square for each distinguished open immersion is cartesian in $\Sh{\XTop}$,
  fixing a finitely presented commutative ring $A$ and an element $a\in A$:
  \[
    \DiagramSquare{
      nw/style = pullback,
      sw = \prn{f^*\Str{\YTop}}^*{A},
      nw = \prn{f^*\Str{\YTop}}^*\prn{A\brk{\Inv{a}}},
      se = \Str{\XTop}^*{A},
      ne = \Str{\XTop}^*\prn{A\brk{\Inv{a}}},
      width = 3cm
    }
  \]

  The role of these cartesian squares can be understood as follows, in the
  internal language of $\Sh{\XTop}$: they state that an element of the pulled
  back local ring $f^*\Str{\YTop}$ is taken by $f^\sharp$ to a unit of
  $\Str{\XTop}$ if and only if it was already a unit in $f^*\Str{\YTop}$ ---
  which is the same as $f^\sharp$ being a local ring map.
\end{para}

\begin{para}
  %
  An open immersion of locally ringed topoi is one of the form
  $\EmbMor[j]{\YTop}{\XTop}$ such that underlying morphism of topoi is an open
  immersion, and $\Spec{\YTop} \cong j^*\Spec{\XTop}$.
  %
\end{para}




\NewDocumentCommand\Prime{}{\mathbf{Prime}}

\subsection{The spectrum of a commutative ring}\nslabel{sec:geometric-spectrum}

\begin{para}\nslabel{para:spec-underlying-space}
  %
  Consider the geometric theory of a ring equipped with a prime filter (the
  complement of a prime ideal); this gives rise to a projection of classifying
  topoi $\Mor{\brk{\Prime}}{\brk{\Ring}}$. In the category of sets, prime
  filters and prime ideals are interchangeable by taking complements; in a
  general topos, they may not coincide and prime filters are the correct notion
  (in an arbitrary topos, one can localize at a prime filter to obtain a local
  ring). The fiber of $\brk{\Prime}$ over a ring $A$ is then the classifying
  topos of prime filters of $A$, the underlying space of the \emph{spectrum} of
  $A$.
  %
\end{para}

\begin{para}\nslabel{para:localization-geom-mor}
  %
  We will exhibit a morphism of topoi $\Mor{\brk{\Prime}}{\brk{\LocRing}}$
  that takes a ring equipped with a prime filter $S \subseteq{A}$ to the
  local ring $A\brk{\Inv{S}}$. Such a morphism is nothing more than a local ring
  defined in the internal type theory of the classifying topos of prime filters!
  The localization at a prime makes sense in any topos, so we simply take the
  generic ring equipped with a prime filter $\prn{U,S}$ of
  $\SET\brk{\Prime}$ to the local ring $U\brk{\Inv{S}}$.
  %
\end{para}

\begin{para}
  The \emph{spectrum} of a ring is the locally ringed topos defined below using \nscref{para:spec-underlying-space,para:localization-geom-mor}.
  \[
    \begin{tikzpicture}[diagram]
      \SpliceDiagramSquare{
        nw/style = pullback,
        nw = \Spec{A},
        ne = \brk{\Prime},
        se = \brk{\Ring},
        sw = \Pt,
        south = A,
        width = 2.5cm,
      }
      \node (nee) [above = 2.5cm of ne] {$\LocRing$};
      \path[->] (ne) edge (nee);
      \path[->,bend left = 20] (nw) edge node [sloped,above] {$\Str{\Spec{A}}$} (nee);
    \end{tikzpicture}
  \]

  We see that this assignment is contravariantly (pseudo)-functorial, arguing
  that a morphism of rings $\Mor{A}{B}$ sends a prime of $B$ to a prime of $A$
  under inverse image. The construction depicted here also appears in
  \citet{johnstone:1977}.
  %
\end{para}

\begin{para}
%
  Another construction of the Zariski spectrum was suggested to the author by
  Mathieu Anel. We may consider the classifying topos $\brk{\mathbf{LocalLoc}}$ of
  the theory of \emph{local localizations}, a subtopos of the cotensor
  $\vvmathbb{2}\pitchfork\brk{\Ring}$ (the classifier of ring maps);
  $\brk{\mathbf{LocalLoc}}$ is equivalent to $\brk{\Prime}$ from
  \nscref{para:spec-underlying-space}, so the spectrum may be constructed
  analogously.
  %
\end{para}



\begin{para}
  %
  Classically, the spectrum of a ring is a locally ringed topological space
  $\Spec{A}$ and the topos we have constructed here is called the \emph{little
  Zariski topos} of $A$. We prefer to speak of our topos as the spectrum, and
  refer to $\Sh{\Spec{A}} = \SET\brk{\Prime\prn{A}}$ as the \emph{little Zariski logos} of $A$.
  %
\end{para}



\subsection{The big Zariski topos of a commutative ring}

\begin{para}\nslabel{para:big-zariski}
  Let $A$ be a commutative ring, i.e.\ a morphism of topoi
  $\Mor{*}{\brk{\Ring}}$; then the \emph{big Zariski topos} of $\Spec{A}$ is the
  locally ringed topos defined by the following \emph{comma square} in the
  category of topoi:
  \[
    \begin{tikzpicture}[diagram]
      \SpliceDiagramSquare{
        nw/style = {pullback=<-},
        nw = \Zar{\Spec{A}},
        ne = *,
        se = \brk{\Ring},
        sw = \brk{\LocRing},
        west = \Str{\Zar{\Spec{A}}},
        south/style = embedding,
        east = A,
        south = i,
        width = 2.5cm,
        west/style = exists,
      }
      \draw[two cell] (ne) to[bend right=10] node [midway] {$$} (sw);
    \end{tikzpicture}
  \]
\end{para}

\begin{para}
  %
  From \nscref{para:big-zariski}, it can be seen that $\Zar{\Spec{A}}$ is the
  classifying topos of \emph{local $A$-algebras}. Let $\XTop$ be a topos; then
  a morphism $\Mor{\XTop}{A}$ consists of a local ring
  $\Mor[B]{\XTop}{\brk{\LocRing}}$ together with a homomorphism of rings
  $\Mor[\alpha]{A_{\vert \XTop}}{i\circ B}$.
  %
  \[
    \begin{tikzpicture}[diagram]
      \SpliceDiagramSquare{
        nw = \Zar{\Spec{A}},
        ne = *,
        se = \brk{\Ring},
        sw = \brk{\LocRing},
        west = \Str{\Zar{\Spec{A}}},
        west/node/style = {mute,upright desc},
        west/style= mute,
        north/style = mute,
        nw/style = {pullback = {<-,mute},mute},
        south/style = embedding,
        east = A,
        south = i,
        width = 2.5cm,
      }
      \node (nww) [above left = of nw] {$\XTop$};
      \path[->,bend right = 30] (nww) edge coordinate [xshift = 1mm, yshift = 1mm] (l-ctr) node [left] {$B$} (sw);
      \path[->,bend left = 30] (nww) edge coordinate [xshift = -1mm, yshift = -1mm] (r-ctr) (ne);
      \path[->,mute] (nww) edge (nw);
      \draw[two cell] (r-ctr) to[bend right=20] node [midway,sloped,below] {$\alpha$} (l-ctr);
    \end{tikzpicture}
  \]
\end{para}

\begin{para}
  \danger
  %
  The big Zariski topos \emph{is not} obtained as a (weak) pullback of topoi:
  this would make $\Zar{\Spec{A}}$ a subtopos of the point, either the punctual topos
  or the empty topos, depending on whether $A$ is a local ring --- since the
  pullback would be the topoi classifying local rings that are isomorphic to
  $A$!
  %
\end{para}



\begin{para}
  %
  Let $\Mor{A}{B}$ be a morphism of rings; we will show that this gives rise to
  a morphism of topoi $\Mor{\Zar{\Spec{B}}}{\Zar{\Spec{A}}}$ using the universal property
  of the comma square. First, consider the pasting of the morphism $\Mor{A}{B}$
  with the 2-cell of the defining comma square for $\Zar{\Spec{B}}$:
  \[
    \begin{tikzpicture}[diagram]
      \node [pullback = <-] (nw) {$\Zar{\Spec{B}}$};
      \node (ne) [right = 3cm of nw] {$\Pt$};
      \node (sw) [below = of nw] {$\brk{\LocRing}$};
      \node (se) [below = of ne] {$\brk{\Ring}$};
      \path[->] (nw) edge coordinate [xshift=4mm] (StrB) node [left] {$\Str{\Zar{\Spec{B}}}$} (sw);
      \path[embedding] (sw) edge node [below] {$i$} (se);
      \path[->] (nw) edge (ne);
      \path[->,bend right = 30] (ne) edge coordinate [xshift = 1mm] (B) node [left] {$B$} (se);
      \path[->,bend left = 30] (ne) edge coordinate [xshift = -1mm] (A) node [right] {$A$} (se);
      \draw[two cell] (A) to[bend right = 30] (B);
      \draw[two cell] (ne) to[bend right=10] (sw);
    \end{tikzpicture}
  \]

  From the composite 2-cell, we have desired morphism of topoi:
  \[
    \begin{tikzpicture}[diagram]
      \SpliceDiagramSquare{
        nw = \Zar{\Spec{A}},
        ne = *,
        se = \brk{\Ring},
        sw = \brk{\LocRing},
        west = \Str{\Zar{\Spec{A}}},
        west/node/style = {upright desc},
        nw/style = {pullback = <-},
        south/style = embedding,
        east = A,
        south = i,
        width = 3.5cm,
      }
      \node (nww) [above left = 2.5cm of nw] {$\Zar{\Spec{B}}$};
      \path[->,bend right = 30] (nww) edge coordinate [xshift = 1mm, yshift = 1mm] (l-ctr) node [left] {$\Str{\Zar{\Spec{B}}}$} (sw);
      \path[->,bend left = 30] (nww) edge coordinate [xshift = -1mm, yshift = -1mm] (r-ctr) (ne);
      \path[->] (nww) edge (nw);
      \draw[two cell] (r-ctr) to[bend right=20] node [midway,sloped,above,] {$\simeq$} (nw);
      \draw[two cell] (nw) to[bend right=20] node [midway,sloped,above,] {$\simeq$} (l-ctr);
      \draw[two cell] (ne) to[bend right=10] (sw);
    \end{tikzpicture}
  \]
  %
\end{para}

\subsection{Affine geometric schemes}

\begin{para}
  %
  An \emph{affine geometric scheme} is a locally ringed space lying in the
  essential image of the pseudofunctor
  $\Mor[\Spec]{\OpCat{\Ring}}{\LocRingedTOP}$ defined in
  \nscref{sec:geometric-spectrum}; explicitly, this an affine geometric scheme is
  a locally ringed topos that is \emph{equivalent} to one of the form
  $\Spec{A}$.  Will write $\Aff$ for the full subcategory of $\LocRingedTOP$
  spanned by affine geometric schemes.

  However, note that $\Spec{A}$ is localic; an equivalence of localic topoi
  corresponds to an \emph{isomorphism} between locales because a preorder has
  no non-trivial isomorphisms.
  %
\end{para}

\subsection{Geometric schemes}

\begin{para}
  %
  We observe, but do not prove, that the restriction of $\Spec$ to affine
  schemes exhibits an equivalence $\OpCat{\Ring}\simeq\Aff$.
  %
\end{para}

%\begin{para}
%  %
%  A \emph{distinguished open immersion} of affine schemes is a morphism of the
%  form $\Mor[D\prn{a}]{\Spec{A\brk{\Inv{a}}}}{\Spec{A}}$ for some element $a \in A$.
%  %
%\end{para}

\NewDocumentCommand\Sub{m}{\mathbf{Sub}\prn{#1}}

\begin{para}\nslabel{para:geometric-scheme}
  %
  Let $\XTop$ be a locally ringed topos. $\XTop$ is a called a \emph{geometric scheme}
  when there exists a set of open immersions $\EmbMor{\XTop_{U_i}}{\XTop}$ such
  that $\bigvee U_i=\top$ and each $\XTop_{U_i}$ is affine; in particular,
  every affine scheme is a scheme. We define $\Sch$ to be the full subcategory
  of $\LocRingedTOP$ spanned by schemes.
  %
\end{para}


\section{The functor of points}

\subsection{The category of functorial spaces}

\begin{para}
  \danger
  %
  One is tempted to use the logos of sheaves on the big Zariski topos
  $\ZTop = \Zar{\Spec{\mathbb{Z}}}$, the classifier of local rings, as a target for the
  functor of points. This is not a good idea, because the full category of
  schemes does not embed fully faithfully into $\Sh{\ZTop}$
  --- only some sufficiently finitary schemes. Instead, we need a logos $\ECat$ whose
  \emph{sheaves} are some kind of generalized schemes. 

  The downside of the correct choice for the logos of functorial spaces is that
  it is unclear what the corresponding topos classifies. While this may be
  aesthetically disappointing, it is worth remembering that $\ECat$ will always
  be viewed as a logos, a category \emph{in which} geometry happens, and not as
  a topos (a geometrical object).
  %
\end{para}

\begin{para}
  %
  The category $\Ring$ of $\UU$-small rings is $\VV$-small.
  %
  Therefore, the category of \emph{(functorial) pre-spaces} $\FunPreSp =
  \Psh[\VV]{\OpCat{\Ring}}$ is a logos, and the Yoneda embedding
  $\EmbMor{\OpCat{\Ring}}{\FunPreSp}$ is fully faithful.
  %
\end{para}

\NewDocumentCommand\FPts{m}{\mathsf{h}\prn{#1}}

\begin{para}
  We define the \emph{functor of points} $\FPts{\XTop}:\FunPreSp$ for a geometric scheme $\XTop$:
  \begin{align*}
    \FPts{\XTop} &: \Mor|{|->}|{A}{\Hom[\Sch]{\Spec{A}}{\XTop}}
  \end{align*}

  The functor of points is itself functorial: it is the nerve corresponding to
  the \emph{dense} embedding $\Mor{\OpCat{\Ring}\simeq\Aff}{\Sch}$, and it is
  consequently fully faithful.
\end{para}

\begin{para}
  %
  While every scheme can be imported into $\FunPreSp$ fully faithfully, the
  gluing of schemes does not correspond to the gluing of their corresponding
  functors of points. This can be resolved by means of \emph{localization}, in
  which we replace $\FunPreSp$ with a quotient logos $\FunSp$ that forces the functors
  of points of certain gluings of schemes to become colimits.
  %
\end{para}

\begin{para}
  %
  Let $\XTop=\Spec{A}$ be a geometric affine scheme; if $\brc{a_i}_{i\in I}$ is a
  finite family of elements of $A$ that generates the unit ideal, then the corresponding family of
  open immersions $\brc{\EmbMor{\Spec{A\brk{\Inv{a_i}}}}{\Spec{A}}\vertsep i\in I}$
  induced by localizing at each $a_i$ is called a \emph{Zariski open cover}
  of $\XTop$. The resulting \emph{Zariski coverage} on $\OpCat{\Ring}$ is written $K$.
  %
\end{para}

\begin{para}
  %
  Let $\XTop = \Spec{A}$ be an affine scheme and let
  $\brc{\EmbMor{\XTop_a}{\XTop}}$ be a Zariski open cover; the data of such an
  open cover is concentrated in the single induced map $\Mor{\Coprod{i}{\FPts{\XTop_{a_i}}}}{\FPts{\XTop}}$ in $\FunPreSp$. We will refer to maps of
  this form as \emph{generating Zariski local covers}.
\end{para}

\begin{para}\nslabel{para:zariski-local-cover}
  %
  The notion of Zariski open covers can be extended from $\Aff$ to $\FunPreSp$,
  to generate exactly the collection of maps in $\FunPreSp$ that we want to
  force to become covers; we say that a natural transformation $\Mor[f]{E}{F}$
  is a \emph{Zariski local cover} when for every $\Spec{A}$-valued point of $F$
  lifts along some generating Zariski local cover through $f$ to a point of
  $E$, as follows:
  %
  \[
    \DiagramSquare{
      nw = \Coprod{i}{\Yo{A\brk{\Inv{a_i}}}},
      width = 3cm,
      north/style = exists,
      sw = \Yo{A},
      se = F,
      ne = E,
      east = f,
      south = x,
      north = \tilde{x}
    }
  \]
\end{para}

\begin{para}\nslabel{para:zariski-local-system}
  The collection of Zariski local covers is a system of local covers in the following sense:
  \begin{enumerate}
    \item Every cover is a Zariski local cover.
    \item Zariski local covers are closed under composition.
    \item If $g\circ f$ is a Zariski local cover, then $g$ is a Zariski local cover.
    \item A morphism $\Mor{E}{F}$ is a Zariski local cover if and only if for each $\Spec{A}$-valued point of $F$, the pullback $\Mor{E\times_{F}\Yo{A}}{\Yo{A}}$ is a Zariski local cover.
  \end{enumerate}
\end{para}


\begin{para}
  %
  A \emph{Zariski dense monomorphism} is a Zarizki local cover that is at the
  same time a monomorphism in $\FunPreSp$. In particular, a \emph{Zariski
  covering sieve} is a Zariski dense monomorphism of the form $U\subseteq\Yo{A}$.
  The dense monomorphisms are maps that will
  become isomorphisms under localization; dense monomorphisms are not the only
  maps that will become isomorphisms, because some non-monomorphisms may become
  isomorphisms as well.
  %
\end{para}

\begin{para}\nslabel{para:generating-local-cover-to-sieve}
  %
  Every generating Zariski local cover
  $\Mor{\Coprod{i}{\Yo{A\brk{a_i}}}}{\Yo{A}}$ generates a covering sieve
  $S_{\brc{a_i}}\subseteq\Yo{A}$ defined as the set of ring morphisms factoring through one
  of the localizations $\Mor|->>|{A}{A\brk{\Inv{a_i}}}$.
\end{para}

\begin{para}\nslabel{para:functorial-space}
  %
  A presheaf $X$ is called a \emph{functorial space} when it is right orthogonal to any
  Zariski dense monomorphism, or (without loss of generality) any Zariski
  covering sieve $\Mor|>->|{U}{\Yo{A}}$.
  %
  \[
    \begin{tikzpicture}[diagram]
      \SpliceDiagramSquare{
        nw = U,
        sw = \Yo{A},
        west/style = {>->},
        ne = X,
        se = \ObjTerm{\FunPreSp},
      }
      \path[exists,->] (sw) edge node [desc] {$\exists!$} (ne);
    \end{tikzpicture}
  \]

  We define the category $\FunSp$ of \emph{functorial spaces} as the full
  subcategory of $\FunPreSp$ spanned by functorial spaces. Moreover, $\FunSp$ is a logos
  by \nscref{para:zariski-local-system}.
  %
\end{para}

\begin{para}
  %
  There is a more concrete way to check that a presheaf $X : \FunPreSp$ is a
  functorial space; for each generating Zariski local cover
  $\Mor{\Coprod{i}{\Yo{A\brk{\Inv{a_i}}}}}{\Yo{A}}$, it is enough for the
  following to be an equalizer diagram:
  \[
    \begin{tikzpicture}[diagram]
      \node (0) {$X\prn{A}$};
      \node (1) [right = 2.5cm of 0] {$\Prod{i}{X\prn{A\brk{\Inv{a_i}}}}$};
      \node (2) [right = 3.5cm of 1] {$\Prod{i,j}{X\prn{A\brk{\Inv{a_{i,j}}}}}$};
      \path[->] (0) edge (1);
      \path[->,shift left =.1cm] (1) edge (2);
      \path[->,shift right =.1cm] (1) edge (2);
    \end{tikzpicture}
  \]

  %% LOL: the following argues that ordinary sheafhood implies this equalizer thing, but I had intended to prove the converse.

%   We fix a matching family $s_\bullet\in {\Prod{i}{X\prn{A\brk{\Inv{a_i}}}}}$
%   that agrees on the overlaps, i.e.\ $s_i$ and $s_j$ restrict to the same
%   section of $X$ at $A\brk{\Inv{a_{i,j}}}$; we must check that there is a
%   unique section of $X$ at $A$ that restricts to each $s_i$. To do so, we will
%   use the orthogonality condition of \nscref{para:functorial-space}, which takes
%   a covering sieve as input.
% 
%   We choose the covering sieve $U = S_{\brc{a_i}}$ from
%   \nscref{para:generating-local-cover-to-sieve}, defining an element
%   $\Mor{U}{X}$; fixing a ring map $\Mor[f = b\circ e_i]{A}{B}$ where
%   $\Mor|->>|[e_i]{A}{A\brk{\Inv{a_i}}}$ and $\Mor[b]{A\brk{\Inv{a_i}}}{B}$, we
%   choose the section $X\prn{b}\prn{s_i}$ of $X$ at $B$. By
%   \nscref{para:functorial-space}, we have a morphism $\Mor{\Yo{A}}{X}$ uniquely
%   factoring the element $\Mor{U}{X}$, and by the Yoneda lemma a corresponding
%   section $\tilde{s}\in X\prn{A}$. 
%   %
%   We must still argue that $\tilde{s}$ is unique in the sense of the equalizer
%   diagram, but this follows from the fact that any other such section would
%   determine an element $\Mor{\Yo{A}}{X}$ that also factors $\Mor{U}{X}$.
%   %
\end{para}

\begin{para}
  %
  Given an affine scheme $\XTop = \Spec{B}$, the representable presheaf $\Yo{B}
  = \FPts{\Spec{B}} : \FunPreSp$ is a functorial space --- this follows almost
  immediately from \nscref{partition-of-unity/para}, because the gluing may be
  used to construct a ring map; consequently, we will refer to any
  representable as a \emph{(functorial) affine scheme}. 

  \begin{proof}
    %
    Suppose we have a matching family of ring maps
    $\Mor[a_i]{B}{A\brk{\Inv{f_i}}}$; we will define a ring map $\Mor[a]{B}{A}$
    using the partition of unity argument.  For each $b$, we have $a_i\prn{b}
    \in A\brk{\Inv{f_i}}$; by \nscref{partition-of-unity/para} these may be
    glued into a single $a\prn{b}$. This assignment is functional because of
    the uniqueness of gluings; it remains to observe that the function
    so-defined is a homomorphism of rings, i.e.\ commutes with all ring
    operations, but this follows from the fact that each $a_i$ is a
    homomorphism of rings.
    %
  \end{proof}
\end{para}

\clearpage

\section{[My plan]}

\begin{enumerate}
  \item Define geometric schemes as locally ringed topoi covered by (geometric) affine schemes

  \item Define big zariski topos, nerve / functor of points of a geometric scheme

  \item Define open immersions into the functor of points, eliminating enough cuts to kill all references to geometric schemes

  \item Define functorial schemes 
\end{enumerate}
