\chapter{Functorial geometry}
\minitoc

%\section{Commutative algebra}



\NewDocumentCommand\Inv{m}{\mindelim{1}{\textstyle\frac{1}{#1}}}
\NewDocumentCommand\Aff{}{\mathbf{Aff}}
\NewDocumentCommand\AffFP{}{\mathbf{Aff}_{\textit{f.p.}}}
\NewDocumentCommand\CRing{}{\mathbf{CRing}}
\NewDocumentCommand\CRingLoc{}{\mathbf{CRing}_{\mathrm{Loc}}}
\NewDocumentCommand\CRingFP{}{\mindelim{1}\mathbf{CRing}_{\textit{f.p.}}}
\NewDocumentCommand\RingedTOP{}{\mathbf{RingedTop}}
\NewDocumentCommand\Spec{g}{\operatorname{Spec}\IfValueT{#1}{#1}}
\NewDocumentCommand\Str{G{\bullet}}{\mathcal{O}_{#1}}
\NewDocumentCommand\GS{g}{\Gamma\IfValueT{#1}{\prn{#1}}}

\section{Commutative algebra and classifying topoi}

We will write $\CRing$ for the category of (commutative) rings and $\CRingLoc$
for the category of \emph{local rings}. Both of these theories are geometric,
and therefore induce classifying topoi $\brk{\CRing},\brk{\CRingLoc}$
respectively.

\section{Affine schemes}

\begin{para}
  %
  An \emph{affine scheme} is the formal dual of a commutative ring, i.e.\ an
  object of the category $\Aff = \OpCat{\CRing}$. 
 % When $A$ is a commutative
 % ring, we will write $\Spec{A}$ for the corresponding affine scheme;
 % conversely, when $X$ is an affine scheme, we write $\Str{X}$ for the
 % corresponding commutative ring.
  %
\end{para}

\begin{para}
  A \emph{distinguished open immersion} of affine schemes is the dual to a
  localization map $\Mor{A}{A\brk{\Inv{f}}}$ of commutative rings for some $f \in A$.
  %in which $Y \cong \Spec{\Str{X}\brk{f^{-1}}}$ for a function $f \in
  %\Str{X}$. 
\end{para}

%\begin{definition}
%  A finitely presented affine scheme is an object of the category $\AffFP = \OpCat{\CRingFP}$.
%\end{definition}


\section{Geometric spaces, affine and general}

We begin with the classic perspective of schemes as locally ringed topoi.

\begin{para}
  %
  A \emph{ringed topos} is a pair $\prn{\XTop,\Str{\XTop}}$ where $\XTop$ is a
  topos and $\Str{\XTop}$ is a \emph{sheaf of rings} in on $\XTop$, i.e.\
  morphism of topoi $\Mor{\XTop}{\brk{\CRing}}$.
%
  We will frequently abuse notation by writing $\XTop$ for the pair
  $\prn{\XTop,\Str{\XTop}}$.
\end{para}

\begin{para}
  The correct notion of morphism between ringed topoi is given by the \emph{lax
  slice} 2-category $\RingedTOP = \LaxSl{\TOP}{\brk{\CRing}}$ whose objects are
  evidently the ringed topoi.
  %
  Consequently, a 1-morphism of ringed topoi
  $\Mor{\prn{\XTop,\Str{\XTop}}}{\prn{\YTop,\Str{\YTop}}}$ is a morphism of
  topoi $\Mor[f]{\XTop}{\YTop}$ together with a morphism of ring objects
  $\Mor[f^\sharp]{f^*\Str{\YTop}}{\Str{\XTop}}$ called the \emph{comorphism}.
  %
\end{para}

\begin{para}
  From the algebraic perspective, the structure sheaf
  $\Mor[\Str{\XTop}]{\XTop}{\brk{\CRing}}$ is a lex and cocontinuous functor
  $\Mor[\Str{\XTop}^*]{\SET\brk{\CRing}}{\Sh{\XTop}}$. Precomposing with the
  Yoneda embedding, one obtains a functor $\Mor{\OpCat{\CRingFP}}{\Sh{\XTop}}$
  that we shall also denote by $\Str{\XTop}^*$.

  Every finitely presented commutative ring can be thought of as a ``context''
  in the language of rings, where there are two distinguished forms of context extension:
  freely adding a new element, and adding an inverse to an existing element.
%
  Letting $A = \brk{x}$ be the free ring generated by a single element, it is
  correct to think of $\Str{\XTop}^*A$ as the sheaf collection of elements of
  $\Str{\XTop}$; likewise, it is correct to think of
  $\Str{\XTop}^*\prn{A\brk{\Inv{x}}}$ as the sheaf of units of $\Str{\XTop}$.
\end{para}

\begin{para}
  %
  A locally ringed topos is a ringed topos $\XTop$ such that $\Str{\XTop}$ is a
  local ring, i.e.\ a ring object such that ``$\Str{\XTop}$ has a unique
  maximal ideal'' holds in the internal logic of $\Sh{\XTop}$.
%
  A morphism of locally ringed topoi is a morphism of ringed topoi
  $\Mor[\prn{f,f^\sharp}]{\XTop}{\YTop}$ such that the induced naturality
  square for each distinguished open immersion is cartesian in $\Sh{\XTop}$,
  fixing a finitely presented commutative ring $A$ and an element $a\in A$:
  \[
    \DiagramSquare{
      nw/style = pullback,
      sw = \prn{f^*\Str{\YTop}}^*{A},
      nw = \prn{f^*\Str{\YTop}}^*\prn{A\brk{\Inv{a}}},
      se = \Str{\XTop}^*{A},
      ne = \Str{\XTop}^*\prn{A\brk{\Inv{a}}},
      width = 3cm
    }
  \]

  The role of these cartesian squares can be understood as follows, in the
  internal language of $\Sh{\XTop}$: they state that an element of the pulled
  back local ring $f^*\Str{\YTop}$ is taken by $f^\sharp$ to a unit of
  $\Str{\XTop}$ if and only if it was already a unit in $f^*\Str{\YTop}$ ---
  which is the same as $f^\sharp$ being a local ring map.
\end{para}

\subsection{The spectrum of a commutative ring}

\begin{para}\label{para:geometric-spectrum}
  Let $A$ be a commutative ring, i.e.\ a morphism of topoi
  $\Mor{*}{\brk{\CRing}}$; then the \emph{geometric spectrum} of $A$ is the
  locally ringed topos defined by the following \emph{comma square} in the
  category of topoi:
  \[
    \begin{tikzpicture}[diagram]
      \SpliceDiagramSquare{
        nw = \Spec{A},
        ne = *,
        se = \brk{\CRing},
        sw = \brk{\CRingLoc},
        west = \Str{\Spec{A}},
        south/style = embedding,
        east = A,
        south = i,
        width = 2.5cm,
        west/style = exists,
      }
      \draw[two cell] (ne) to[bend right=10] node [midway] {$$} (sw);
    \end{tikzpicture}
  \]

\end{para}

\begin{para}
  %
  From \cref{para:geometric-spectrum}, it can be seen that $\Spec{A}$ is the
  classifying topos of \emph{local $A$-algebras}. Let $\XTop$ be a topos; then
  a morphism $\Mor{\XTop}{A}$ consists of a local ring
  $\Mor[B]{\XTop}{\brk{\CRingLoc}}$ together with a homomorphism of rings
  $\Mor[\alpha]{A_{\vert \XTop}}{i\circ B}$.
  %
  \[
    \begin{tikzpicture}[diagram]
      \SpliceDiagramSquare{
        nw = \Spec{A},
        ne = *,
        se = \brk{\CRing},
        sw = \brk{\CRingLoc},
        west = \Str{\Spec{A}},
        west/node/style = {mute,upright desc},
        west/style= mute,
        north/style = mute,
        nw/style = mute,
        south/style = embedding,
        east = A,
        south = i,
        width = 2.5cm,
      }
      \node (nww) [above left = of nw] {$\XTop$};
      \path[->,bend right = 30] (nww) edge coordinate [xshift = 1mm, yshift = 1mm] (l-ctr) node [left] {$B$} (sw);
      \path[->,bend left = 30] (nww) edge coordinate [xshift = -1mm, yshift = -1mm] (r-ctr) (ne);
      \path[->,mute] (nww) edge (nw);
      \draw[two cell] (r-ctr) to[bend left=30] node [midway,sloped,below] {$\alpha$} (l-ctr);
    \end{tikzpicture}
  \]
\end{para}

\begin{para}
  \danger
  %
  The geometric spectrum \emph{is not} obtained as a (weak) pullback of topoi:
  this would make $\Spec{A}$ a subtopos of the point, either the punctual topos
  or the empty topos, depending on whether $A$ is a local ring --- since the
  pullback would be the topoi classifying local rings that are isomorphic to
  $A$!
  %
\end{para}



\section{[My plan]}

\begin{enumerate}
  \item Define geometric schemes as locally ringed topoi covered by (geometric) affine schemes

  \item Define big zariski topos, nerve / functor of points of a geometric scheme

  \item Define open immersions into the functor of points, eliminating enough cuts to kill all references to geometric schemes

  \item Define functorial schemes 
\end{enumerate}

%then the pair $\prn{\Pt,A}$ is obviously a
%ringed topos, considering the fact that $\SET = \Sh{\Pt}$,    

%We will write $\AFF = \OpCat{\CRing}$ for the category of \emph{affine
%schemes}, the formal dual to the category of commutative rings.

